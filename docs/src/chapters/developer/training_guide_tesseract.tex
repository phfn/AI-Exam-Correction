\author{Bautrelle Fotso}
\graphicspath{ {./src/chapters/developer/media/} }

\chapter{Build a new model using tesseract} (\cite{[11]})
This are detailled steps to be executed to get a model (font) ready to be used for text recognition.
Items are step instructions and subitems are command lines to execute in the console.

\begin{enumerate}
	\item converted to a ".tif" format because it is required from tesseract to easily predict and build boxes
	\begin{itemize}
	  \item Convert –density 300 Bild.png lang.font.exp0.tif
	\end{itemize}
	\item sent to tesseract in order to recognise characters and build its respective boxes
	\item tesseract then predict the word image and make boxes for each
	\begin{itemize}
		\item tesseract lang.font.exp0.tif lang.font.exp0 batch.nochop makebox
	  \end{itemize}
	\item the output is a boxfile. with the extension ".box"
	\item The two files are saved in the same ordner.
	\item the JTessBoxEditor is opened and only the .tif file is loaded inside, NOT the box file. 
	\item The Editors recognizes also the box file from the name and load it also.  
	\item The labelling is performed, the coordinates of the boxes are corrected and the changes are saved.
	\item Both, box and .tif file are sent again to tesseract 
	\item The boxes are now trained. The output is a file with the extension ".tr"
	\begin{itemize}
		\item tesseract lang.font.exp0.tif lang.font.exp0 nobatch box.train
	  \end{itemize}
	\item Characters are extracted from each .tr file 
	\begin{itemize}
		\item unicharset\_extractor lang.font.exp0.box
	  \end{itemize}
	\item create a file named font\_property and insert the characteristics of the font
	\begin{itemize}
		\item touch lang.font\_properties > 0 0 0 0 0 because handwritings has no italic, bold, fixed, serif or fraktur
	  \end{itemize}
	\item create clusters and prototypes
	\begin{itemize}
		\item mftraining -F lang.font\_properties -U unicharset lang.font.exp0.tr
	  \end{itemize}
	\item Generate the feature files .normproto 
	\begin{itemize}
		\item cntraining -F lang.font\_properties -U lang.font.exp0.tr
	  \end{itemize}
    \item rename the following output files by adding the language name as follow: lang.unicharset, lang.inttemp, lang.shapetable and lang.normpoto
    \item combine all data having the language name "lang." at the beginning together.
	\begin{itemize}
		\item combine\_tessdata lang.
	  \end{itemize}
 \end{enumerate}

After the new font have been trained, it is saved to the tesseract languages folder "tessdata".
The path where it can be found starts with /usr/share/ or /usr/local/share.

All the handwriting documents used to deploy the SWTP-AI dataset as well as the tesseract model built in this project
can be found in the project source(backend/docs).
