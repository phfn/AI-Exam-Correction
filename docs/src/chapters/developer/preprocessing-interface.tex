\author{Tadd\"aus Nauheimer}
\chapter{Preprocessing Interface}
In this chapter we will talk about the preprocessing interface and the points calculation for the tasks that is happening in it.
The preprocessing interface is the connection between the web-backend and the actual ocr and shape detection.
In it the entire exam container will be processed into the single tasks which then will have the expected/actuall answers set.
After the processing of the tasks and detection of answers the task will be put into the exam container again.

The current tasks that can be send to the preprocessing backend are:

\begin{itemize}
	\item Single Choice Checkboxes
	\item Multiple Choice Checkboxes (but not yet connected to the frontend)
	\item Text (with numbers)
	\item Text without numbers
	\item Number
\end{itemize}


\section{Points Calculation}
Another step taken in this is the point calculation.
For this there are a few relevant variables from the taks class that are used here.

\begin{itemize}
	\item max\_points
	\item deduction\_per\_error
	\item points
	\item actual\_answer/expected\_answer
\end{itemize}

The points calculation is relativly basic.
The system compares the actual\_answer and the expected\_answer. 
Whenever a difference is noticed the amount given in deduction\_per\_error is substracted from max\_points.
If the points amount reaches 0 or the length of the expected answer is reached the value returns the points.
This amount will be set in the points variable from the Task class.
